%% Dokumenteinstellungen
\documentclass{template}

%%Dokument
\begin{document}

% Ersten Seiten
\frontmatter

% Die eigentlichen Arbeiten
\chapter{Securitymechanismen und -Analyse der ONOS-Plattform}
\chapterauthor{Aaron Miehle}

\Kurzfassung{
Hier soll ein Abstract stehen. Hier soll ein Abstract stehen. Hier soll ein Abstract stehen. Hier soll ein Abstract stehen. Hier soll ein Abstract stehen. Hier soll ein Abstract stehen. Hier soll ein Abstract stehen. Hier soll ein Abstract stehen. Hier soll ein Abstract stehen. Hier soll ein Abstract stehen. Hier soll ein Abstract stehen. Hier soll ein Abstract stehen. Hier soll ein Abstract stehen. Hier soll ein Abstract stehen. Hier soll ein Abstract stehen. Hier soll ein Abstract stehen. Hier soll ein Abstract stehen. Hier soll ein Abstract stehen. Hier soll ein Abstract stehen. Hier soll ein Abstract stehen. Hier soll ein Abstract stehen. Hier soll ein Abstract stehen. Hier soll ein Abstract stehen. Hier soll ein Abstract stehen. Hier soll ein Abstract stehen. Hier soll ein Abstract stehen. Hier soll ein Abstract stehen. Hier soll ein Abstract stehen.

}

\newpage

%%%% Und hier fängt die Seminararbeit an


\section{Einleitung}
Software-defined Networking (SDN) beschreibt eine Architektur große Netzwerke zu erstellen, zu strukturieren und zu betreiben. Merkmale dieser Architektur sind die abgekoppelte Netzwerkkontrolle und Datenebene sowie die Programmierbarkeit über das Kontrollpanel.~\cite{Xia_a_survery_on_sdn_2015}

Eine darauf basierende Technologie ist das Open Network Operating System (ONOS), welcher der am meisten genutzte SDN-Controller für die Erstellung von SDN-Lösungen darstellt. Hierbei unterstützt ONOS sowohl  ~\cite{opennetworkingOpenNetwork}

\subsection{Motivation}

Trotz der Vorteile, die die ONOS-Plattform mit sich bringt, ist der Aspekt der Sicherheit in der Nutzung ein essentieller Faktor, der weiterhin zu analysieren ist. 
\subsection{Zielsetzung}

\newpage

\section{Theoretische Grundlagen}

\subsection{Überblick über die ONOS-Plattform}
\cite{ohri_security_2023}
\subsection{SDN-Architektur und Einordnung von ONOS}

\newpage

\section{Sicherheitsmaßnahmen in ONOS}

\newpage
\section{Schwachstellen in ONOS}

\newpage
\section{Einschätzung der sicheren Nutzung von ONOS}

\newpage
\section{Fazit}


%%% Ab hier nichts ändern
\pagebreak
\renewcommand\bibname{Literaturverzeichnis}
\bibliographystyle{plainnat}
\bibliography{01-arbeit}

\include{02-arbeit}

% Jedes Thema hat eine Nummer. Diese Nummer verwenden, um die eigentliche Arbeit zu schreiben, also Thema 1 hat 01-arbeit.tex und 01-arbeit.bib.
% Da jeder ein Chapter schreibt, sollte der Text jeweils darunter folgen.
% Abkürzungsverzeichnis und passende Packages sind bei diesem Ansatz leider schwierig umzusetzen.
% Zum Kompilieren entweder bash compile.sh oder entsprechend die Befehle ausführen. Da mit chapterbib gearbeitet wird, muss jedes aux file mit bibtex kompiliert werden.

\end{document}
